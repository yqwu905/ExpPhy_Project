\documentclass{article}
\usepackage[dvipsnames, svgnames, x11names]{xcolor}
\usepackage{ctex}
\usepackage{cite}
\usepackage{siunitx}
\usepackage{enumerate}
\usepackage{amsfonts}
\usepackage{amsmath}
\usepackage{physics}
\usepackage{graphicx}
\usepackage{hyperref}
\usepackage[most]{tcolorbox}
\usepackage{subfigure}
\usepackage{float}
\usepackage{amssymb}

\begin{document}
\begin{titlepage}
  %\clearpage
  \thispagestyle{empty}
  \centering
  \vspace{1cm}

  % Titles
  % Information about the University
  {\
	  \textsc{武汉大学 2021-2022学年 综合物理实验5}
  }
  \vspace{1.5 cm}

  \rule{\linewidth}{2mm} \\[0.8cm]
  { \LARGE \sc 基于神经网络的风格迁移}\\[0.55cm]
  \rule{\linewidth}{0.6mm} \\[2.4cm]

  \hspace{1cm}
  \begin{tabular}{l p{6cm}}
          \textbf{Name} & \textbf{吴远清}\\[10pt]
          \textbf{Department} & 弘毅学堂 \\[10pt]
          \textbf{Date} & \today \\
  \end{tabular}

  %\vfill
  \vspace{2cm}
  % Light logo and Dark logo
\begin{center}
\includegraphics[width=4.5cm]{image/school_tag.png}
\end{center}
  \begin{center}
\includegraphics[width=4.5cm]{image/school_name.jpg}
\end{center}
  \vspace{0.5cm}
  %\pagebreak
  \global\let\newpagegood\newpage
\global\let\newpage\relax
\end{titlepage}
\global\let\newpage\newpagegood
%%%%%%%%%%%%%%%%%%%%%%%%%%%%%%%%%%%%%%%%%%%%%%%%%%%%%%%%%%%%%%%%%%%%%%%%%%%%%%%%%%%%%%%%%%%%
%%%%%%%%%%%%%%%%%%%%%%%%%%%%%%%%%%%%%%%%%%%%%%%%%%%%%%%%%%%%%%%%%%%%%%%%%%%%%%%%%%%%%%%%%%%%
%%%%% Text body starts here!
%\clearpage
%\mainmatter
\setcounter{page}{1}

    % \maketitle
    % \centerline{\includegraphics[width=4.5cm]{image/school_tag.png}}
    \clearpage
    \tableofcontents
    \clearpage
	\section{简介}
	图像风格迁移是指对于两张给定的图片:
	一张内容图片(c)和一张风格图片(s),
	我们训练一个神经网络将c以s的'风格'重绘.\\
	目前有多种图像风格迁移的实现方式,
	例如可以通过基于生成对抗网络(Generative Adversarial Net, GAN)来实现,
	也可以通过传统的卷积神经网络(Convolutional Neural Network, CNN),
	最近的研究成果也有利用较新的Transformer模型来实现图像风格迁移的.\\
	在本实验报告中, 我将介绍基于CNN实现图像风格迁移的理论和技术细节,
	并给出一个基于PyTorch的实现, 展示一些在不同的内容图片与风格图片之间得到的结果,
	最后讨论一下模型中的参数设置与存在的一些问题, 并进行总结.

	\section{理论}
	利用CNN实现图像风格迁移的主要思路是, 输入{\bf 内容图像c}和{\bf 风格图像s}, 为了将s的风格迁移到c上, 我们需要分别提取出内容图像和风格图像的特征图,
	之后生成一个随机噪声图片x, 分别计算其内容特征图和风格特征图, 计算出x的特征图与c和s的特征图的损失函数, 并调整图像x, 使得x的风格特征图接近于s, 而x
	的内容特征图接近于c.\\
	\subsection{损失函数}
	\bibliographystyle{unsrt}
	\bibliography{reference}
\end{document}
